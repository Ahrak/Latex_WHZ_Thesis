\vfill
\chapter*{Autorenreferat}
\addcontentsline{toc}{chapter}{Autorenreferat}

%\mynote{Folgender Text ist nur Platzhalter! Hier muss noch eine richtige Kurzfassung hin!}

Die Arbeit beschäftigt sich mit der Erstellung eines CRUD-Frameworks innerhalb einer Java Enterprise Umgebung.
Ziel der Arbeit ist der Entwurf und die Umsetzung einer dynamische Benutzeroberfläche zur Manipulation von Entitäten und Beziehungen auf Basis des Javascript Frameworks ExtJS.
Die erstellte Oberfläche passt sich dabei dynamisch an die jeweilige Entität mit zuvor erhaltener Informationen an. 
Diese Informationen werden serversetig mit Hilfe von Annotationen in den entsprechenden Klassen hinterlegt und ausgewertet.
Die Kommunikationsschnittstelle des Servers bildet eine Implementierung der Rest-Architektur über das HTTP-Protokoll.
Für die Übertragung der Entitätsdaten werden die Objekte der entsprechenden EJB-Klassen in eine vereinfachte Form
überführt. 
Nach der Manipulation der Daten werden diese Daten serverseitig wieder in die entsprechenden Fachklassen transformiert und persistiert.
Die Ergebnisse der Arbeit dienen als Grundlage für die Entwicklung eines quelloffenen Frameworks für die Entwicklung von webbasierten Anwendungen.
%Weitere wichtige Elemente der Arbeit sind die Behandlung 