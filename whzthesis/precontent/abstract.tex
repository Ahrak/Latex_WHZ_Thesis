\chapter{Abstract}

\begin{english} %switch to English language rules
This should be a 1-page (maximum) summary of your work in English.
%und hier geht dann das Abstract weiter...
\end{english}

Im englischen Abstract sollte inhaltlich das Gleiche
stehen wie in der deutschen Kurzfassung. Versuchen Sie daher, die
Kurzfassung pr�\-zise umzusetzen, ohne aber dabei Wort f�r Wort zu
�bersetzen. Beachten Sie bei der �bersetzung, dass gewisse
Redewendungen aus dem Deutschen im Englischen kein Pendant haben
oder v�llig anders formuliert werden m�ssen und dass die
Satzstellung im Englischen sich (bekanntlich) vom Deutschen stark
unterscheidet (mehr dazu in Abschn.\ \ref{sec:englisch}). Es
empfiehlt sich �brigens -- auch bei h�chstem Vertrauen in die
pers�nlichen Englischkenntnisse -- eine kundige Person f�r das
"`proof reading"' zu engagieren.

Die richtige �bersetzung f�r "`Diplomarbeit"' ist �brigens
schlicht \emph{thesis}, allenfalls  "`diploma thesis"' oder "`Master's thesis"', auf keinen Fall aber "`diploma work"' oder gar "`dissertation"'. 
F�r "`Bachelorarbeit"' ist wohl "`Bachelor thesis"' die passende �bersetzung. 

�brigens sollte f�r diesen Abschnitt die \emph{Spracheinstellung} in \latex\ von Deutsch
auf Englisch umgeschaltet werden, um die richtige Form der
Silbentrennung zu erhalten, die richtigen Anf�hrungszeichen muss
man allerdings selbst setzen %
(s.\ dazu Abschnitt \ref{sec:sprachumschaltung} %
bzw.\ \ref{sec:anfuehrungszeichen}).
