\chapter{Chronologische Liste der Änderungen}


\begin{sloppypar}
\begin{description}
%
\item[2002/01/07]
\verb!\newfloat{program}! repariert (auch ohne Chapter). Dank an Werner Bailer!
%
\item[2002/03/06]
Copyright-Notice an internat.\ Standard angepasst. Dank an Karin Kosina!
%
\item[2002/07/28]
"`Studiengang"' $\rightarrow$ "`Diplomstudiengang"'
%
\item[2003/08/24]
Neuer Macro: \verb!\Messbox{breite}{hoehe}! -- zur Kontrolle der 
Druckgröße ohne PS-Datei. Erweiterungen für Bakkalaureatsarbeiten
%
\item[2005/04/09]
Diverse Korrekturen: Captions von Tabellen nach oben gesetzt. 
\texttt{caption} auf neue Versionen adaptiert.
\texttt{subfigure} wird nicht mehr verwendet
%
\item[2006/01/20]
Adaptiert zur Verwendung als Praktikumsbericht 
(2.\ Bakk.-Arbeit)
%
\item[2006/03/24]
Fehler in \verb!\erklaerung! beseitigt (Dank an David Schwingenschlögl)
%
\item[2006/04/06]
Verwendung von T1-Fontencoding zur besseren Silbentrennung bei 
Umlauten etc.
%
\item[2006/06/21]
Neu: Bachelorstudiengang / Masterstudiengang.
Literaturverweise auf Bakk-Arbeiten.
\texttt{upquote.sty} eliminiert (Problem mit TS1-Kodierung).
Verwende Komma (statt Punkt) als Trennzeichen in Dezimalzahlen.
%
\item[2006/09/14]
Anmerkungen zum Thema Plagiarismus.
%
\item[2007/07/16]
Ergänzungen für Code-Listings (listings) und Algorithmen 
(\texttt{algorithmicx}).
BiBTeX-Datei aufgeräumt, Verwendung der Literaturformate 
verbessert.
Komma (statt Punkt) als Trennzeichen in Dezimalzahlen wieder 
entfernt.
Verwendung der \texttt{ae}-Fonts eliminiert (\texttt{cm-super} Fonts müssen 
installiert sein, ab MikTeX 2.5). 
Beispiel für Ersetzung in EPS-Dateien mit \texttt{psfrag}.
%
\item[2007/10/04]
Version 5.90: Das Laden der Pakete \verb!inputenc! (Option \texttt{latin}) und 
\verb!graphicx! (Option \texttt{dvips})
aus der Hauptdatei in die \texttt{sty}-Datei übertragen; \texttt{upquote} funktioniert nun.
Paket \texttt{eurosym} ergänzt für Euro-Symbol (Anregung von Andreas 
Doubrava).
Problem mit \texttt{color}-package repariert (gerasterter PDF-Ausdruck).
Hinweise bzgl.\ Literatur ergänzt (\texttt{month}, \texttt{edition}),
BibTeX-Datei gesäubert.
Hinweis zum Einfügen von vertikalem Abstand zwischen Absätzen.
Mathematik aufgeräumt, Verwendung von \texttt{amsmath}, 
Fallunterscheidungen.
Diverse Änderungen bei Tabellen und Programmkode.
Beispiele für BibTeX-Angaben von Spezialquellen: Audio-CDs, 
Videos, Filme. Einbinden von Dateien mit \verb!\include{..}!
Neue Datei: \verb!_SimpleReport.tex! für kurze Reports (Projekte etc.).
%
\item[2007/11/11]
Version 5.91: Hinweise zur Einstellung der Output-Profile in
TexNicCenter, Inverse Search Einstellung in YAP im Anhang.
%
\item[2008/04/01]
Version 6.00beta Kompletter Umbau!
Auslagerung der Doku\-menten-relevanten Teile in eine eigene 
\emph{class}-Datei (\texttt{hgbthesis.cls}) mit Optionen.
Die neue Style-Datei \texttt{hgb.sty} ist nun unabhängig vom 
Dokumententyp und nicht mehr kompatibel mit älteren Versionen!
Die Liste der Änderungen ist jetzt in der Datei \verb!_ChangeLog.tex!
(DIESE Datei) und diese wird im Anhang eingebunden.
Heading-Style auf Sans Serif geändert (ohne grausliche "`Caps"').
%
\item[2008/05/22]
Neue Vorlage für Technical Reports (Klasse \texttt{hgbreport.cls}).
Spracheinstellung nunmehr mit \texttt{babel}-Paket, Hauptsprache
des Dokuments kann als Option der Klasse angegeben werden.
Sprachumschaltung innerhalb des Dokuments funktioniert nun
richtig. Mit der Sprachoption \texttt{german} wird automatisch die neue deutsche 
Orthographie (\texttt{ngerman}) verwendet.
\texttt{babelbib} wird zur Formatierung des Literaturverzeichnisses
verwendet (neue BibTeX-Style-Optionen!).
Header werden nunmehr mit \texttt{fancyhdr}-Paket erzeugt.
Versionsnummerierung von \texttt{.cls} und \texttt{.sty} Files wird beendet 
(ab jetzt gilt: \emph{Datum} = \emph{Version}). 
%
\item[2008/06/10]
Neues Listing-Environment \texttt{PhpCode}; bei allen Listing-Eviron\-ments ist nun 
\texttt{mathescape=false} (kein Math-Mode nach \verb!$!). 
Bug bei Sprachumschaltung auf \texttt{ngerman} beseitigt.
%
\item[2008/08/15]
Diverse Kleinigkeiten in Literaturangaben überarbeitet (Dank an Norbert Wenzel), Spracheinstellung vereinheitlicht, Umlaute in \texttt{.bib}-Datei ersetzt.
%
\item[2008/10/15] 
Zusätzliche Hinweise zur MikTeX-Installation (Windows) sowie LaTeX unter Mac OS~X und Linux.
Liste der Abkürzungen ergänzt.%
\item[2008/11/15] 
Diverse Schreibfehler korrigiert (Dank an Silvia Fuchshuber). Hinweis auf 
\texttt{sloppypar}-Umgebung.
%
\item[2008/12/09] 
BibTeX-Tools: neuer Hinweis auf JabRef ergänzt, BibEdit entfernt (ist nicht mehr auffindbar).
%
\item[2009/02/09]
\texttt{hgb.sty}: Option "`\texttt{spaces}"' zu \texttt{url}-Package ergänzt (ermöglicht gezielten Zeilenumbruch in URLs). 
Im allgemeinen Setup für \texttt{listings}: \texttt{keepspaces=true};
Obsoletes Environment \texttt{sourcecode} deaktiviert.
Escape-Mode für \texttt{LaTeXCode}-Umgebung geändert.
\verb!_DaBa.tex!: Hinweis auf die Verwendung von \verb!\urldef! für die Angabe von URLs in Captions. \texttt{diplom} (statt \texttt{master}) als Standard-Dokumententyp in \verb!_DaBa.tex! ("`Diplomarbeit"'). Neuer Abschnitt zum Umgang mit ``Quellenangaben in Captions''.
\texttt{literatur.bib}: alle URLs (bisher in \texttt{note}-Einträgen) auf \verb!url={..}! geändert.
%
\item[2009/04/14]
Hinweis zum Einfügen einfacher Anführungszeichen ergänzt.
%
\item[2009/07/18]
Literaturangaben korrigiert und ergänzt.
%
\item[2009/11/27]
Experimentelle Version: Massive Änderungen, Umstieg auf \texttt{pdflatex}.
%
\item[2010/06/15]
Erstes Release der neuen Version mit \texttt{pdflatex}.
\item[2010/06/23]
Konflikt zwischen \texttt{pdfsync}-Package und \texttt{array}-Package (wird relativ häufig benutzt) durch \verb!\RequirePackage[novbox]{pdfsync}! behoben.
Seitenunterkante durch \verb!\flushbottom! fixiert,
variablen Absatzzwischenraum reduziert.
%
\end{description}

\end{sloppypar}

%\section*{To Do} 
%\begin{itemize}
%\item Literaturempfehlungen zum Schreiben von Diplomarbeiten
%\item Hinweise für Literatursuche (Bibliotheksverbund, CiteSeer,...)
%\end{itemize}



