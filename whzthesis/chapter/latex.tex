\chapter{\latex Grundlagen}
\label{cha:ArbeitenMitLatex}


\section{Grundstruktur}

\latex benutzt  für die Erstellung von Dokumenten eine Beschreibungsprache (engl. markup language). 
Diese Art der Dokumentenbeschreibung verhält sich Ähnlich wie die weit verbreitete markup language HTML zur Darstellung von Webseiten.
Die Beschreibenden Elemente und der Text werden mit Hilfe der LaTeX Implemtierung in die Dokumentenform konvertiert.
Das minimalistischste Beispiel für Latex sieht dabei wie folgt aus: \\

\begin{lstlisting}
\documentclass{article}
\begin{document}
  Hello world!
\end{document}
\end{lstlisting}

Mit dem documentclass-Befehl wird die Dokumentenklasse gewählt, welche die meisten Grundeinstellung für die Dokumentendarstellung übernimmt \bzw festlegt.
Danach können weitere gloabale Konfigurationen folgen \zB dss einbinden von weiteren Paketen über das Schlüsselwort \verb!\usepackage{paketname}!
Innerhalb der document-Umgebung wird dann der entsprechende Text mit den jeweiligen Formatierungen geschrieben.




\section{Schrift}

\subsection{Schriftfamilien}
%
\begin{quote}
\begin{tabular}{lcl}
\textrm{Roman}      & & \verb!\textrm{Roman}!\\
\textsf{Sans Serif} & & \verb!\textsf{Sans Serif}!\\
\texttt{Typewriter} & & \verb!\texttt{Typewriter}!\\
\end{tabular}
\end{quote}
%

\subsection{Schriftschnitte}
%
\begin{quote}
\begin{tabular}{lcl}
\textit{Kursiv}     & & \verb!\textit{Kursiv}!\\
\textsl{Geneigt}	& & \verb!\textsl{Geneigt}!\\
\textsc{Small Caps} & & \verb!\textsc{Small Caps}!\\
\textup{Aufrecht}   & & \verb!\textup{Aufrecht}!\\
\end{tabular}
\end{quote}
%

\subsection{Schriftserien}
%
\begin{quote}
\begin{tabular}{lcl}
\textbf{Bold}       & & \verb!\textbf{Bold}!\\
\textsl{Geneigt}	& & \verb!\textsl{Geneigt}!\\
\textsc{Small Caps} & & \verb!\textsc{Small Caps}!\\
\textup{Aufrecht}   & & \verb!\textup{Aufrecht}!\\
\end{tabular}
\end{quote}
%

\section{Textgliederung}
\label{sec:ueberschriften}

%
\begin{quote}
\verb!\part{!\texttt{\em Titel}\verb!}!%
\footnote{\texttt{part} ist für die Gliederung eines
größeren Werks in mehrere Teile vorgesehen und wird üblicherweise
bei einer Diplomarbeit (und auch in diesem Dokument) nicht
verwendet.}
\newline%
\verb!\chapter{!\texttt{\em Titel}\verb!}! \newline%
\verb!\section{!\texttt{\em Titel}\verb!}! \newline%
\verb!\subsection{!\texttt{\em Titel}\verb!}! \newline%
\verb!\subsubsection{!\texttt{\em Titel}\verb!}! \newline%
\verb!\paragraph{!\texttt{\em Titel}\verb!}! \newline%
\verb!\subparagraph{!\texttt{\em Titel}\verb!}!
\end{quote}
%

\paragraph{Häufiger Fehler:} Bei \verb!\paragraph{}! und
\verb!\subparagraph{}! läuft -- wie in diesem Absatz zu sehen --
der dem Titel folgende Text ohne Umbruch in der selben Zeile
weiter, weshalb man im Titel auf eine passende Punktuation (hier
\zB\ \underline{\texttt{:}}) achten sollte. Der horizontale Abstand
nach dem Titel allein würde diesen als Überschrift nicht erkennbar
machen.


\subsection{Listen}

Listen sind ein beliebtes Mittel zur Textstrukturierung. In
\latex\ sind -- ähnlich wie in HTML -- drei Arten von formatierten
Listen verfügbar: ungeordnete Auflistung ("`Knödelliste"'),
geordnete Auflistung (Aufzählung) und Beschreibungsliste
(Description):
%
\begin{verbatim}
    \begin{itemize}     ... \end{itemize}
    \begin{enumerate}   ... \end{enumerate}
    \begin{description} ... \end{description}
\end{verbatim}
%
Listeneinträge werden jeweils mit \verb!\item! markiert, bei {\tt
description}-Listen mit \verb!\item[!\texttt{\em titel}\verb!]!. Listen
können ineinander verschachtelt werden, wobei sich bei {\tt
itemize}- und \texttt{enumerate}-Listen die Aufzählungszeichen mit
der Schachtelungstiefe ändern (Details dazu in der
\latex-Dokumentation).


\subsection{Absatzformatierung und Zeilenabstand}

Diplomarbeiten werden -- wie Bücher -- in der Regel einspaltig und
im Blocksatz formatiert, was für den Fließtext wegen der großen
Zeilenlänge vorteilhaft ist. Innerhalb von Tabellen kommt es
wegen der geringen Spaltenbreite jedoch häufig zu Problemen mit
Abteilungen und Blocksatz, weshalb man dort ohne schlechtes
Gewissen zum Flattersatz ("`ragged right"') greifen sollte (wie
\zB\ in Tab.~\ref{tab:synthesis-techniques} auf Seite
\pageref{tab:synthesis-techniques}).

\subsection{Fußnoten}
Fußnoten können in \latex\ an beinahe jeder beliebigen Stelle,
jedenfalls aber in normalen Absätzen, durch die Anweisung
%
\begin{quote}
\verb!\footnote{!\texttt{\em Fußnotentext}\verb!}!
\end{quote}
%
gesetzt werden. Zwischen der \verb!\footnote!-Marke und dem davor
liegenden Text sollte grundsätzlich \emph{kein Leerzeichen} entstehen (eventuelle
Zeilen\-um\-brüche mit \verb!%! auskommentieren).
Die Nummerierung und Platzierung der Fußnoten
erfolgt automatisch, sehr große Fußnoten werden notfalls sogar auf
zwei aufeinanderfolgende Seiten umgebrochen.



\subsection{Definierte Abkürzungen}


\begin{table}
\caption{In \texttt{whz.sty} definierte Abkürzungsmakros.}
\label{tab:abkuerzungen}
\centering
\begin{tabular}{llp{2cm}ll}
\hline
    \verb+\bzw+        & \bzw   & &  \verb+\ua+         & \ua \\
    \verb+\bzgl+       & \bzgl  & &  \verb+\Ua+         & \Ua \\
    \verb+\ca+         & \ca    & &  \verb+\uae+        & \uae \\
    \verb+\dah+        & \dah   & &  \verb+\usw+        & \usw \\
    \verb+\Dah+        & \Dah   & &  \verb+\uva+        & \uva \\
    \verb+\ds+         & \ds    & &  \verb+\uvm+        & \uvm \\
    \verb+\evtl+       & \evtl  & &  \verb+\va+         & \va \\
    \verb+\ia+         & \ia    & &  \verb+\vgl+        & \vgl \\
    \verb+\sa+         & \sa    & &  \verb+\zB+         & \zB \\
    \verb+\so+         & \so    & &  \verb+\ZB+         & \ZB \\
    \verb+\su+         & \su    & &                     &     \\
\hline
\end{tabular}
\end{table}


