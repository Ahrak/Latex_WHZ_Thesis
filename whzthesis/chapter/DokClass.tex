\chapter{Die Dokumentenklasse}


\section{Die Klasse}
\label{sec:class_class}
%=============================================

Die Dokumentenklasse \textit{whzthesis.cls} ist eine \latex-Vorlage für das Erstellen von wissenschaftlichen Arbeiten im Fachbereich Informatik an der \abk{WHZ}.
Die Klasse wurde nach den Richtlinien für das Erstellen von Bachelor-arbeiten im Studiengang Informatik erstellt (\siehe{chap:richtlinien}). 
Die Grundlage für die Dokumentenklasse bildete die Vorlage "`LaTeX-Vorlage für Hochschularbeiten"' von Wilhelm Burger (FH-Hagenberg)\cite{BURG11}. 


\section{Verwenden der Dokumentenklasse}
\label{sec:class_use}
%=============================================

\subsection{Angabe der Klasse}
%------------------------
Um die Vorlage \bzw die Dokumentenklasse zu benutzen müssen die beiden Dateien \textit{whzthesis.cls} und \textit{whz.sty} im gleichem Ordner wie
das Hauptdokumenten liegen. Die Verwendung erfolgt über den \latex-Befehl \textit{$\setminus$documentclass[<options>]\{whzthesis\}}.

\subsection{Minimalbeispiel}
%---------------------------
\lstset{language=[LaTeX]TeX}
\begin{lstlisting}[caption=Beispiel für die Konfiguration der Dokumentenklasse]
\documentclass[
	,bachelor		% type
	,schwarz		% prof
	,selbsterkl		% Selbstaendigkeitserklaerung anzeigen
	,glossar		% Glossar	
	,abbverz		% Abbildungsverzeichnis
]{whzthesis}
\titlename{Meine Wissenschafliche Arbeit}
\authorname{Hans Wurst}
\gebdatum{23. Januar 1983}
\gebort{Musterstadt}
\abgabedatum{27. Mai 2011}
\semester{Sommersemester 2011} 
\betreuerUnternehmen{Dipl. Inf. Max Mustermann, Muster AG, Musterhausen}
\begin{document} ...\end{document}


\end{lstlisting}


\section{Optionen}
\label{sec:class_opts}
%=============================================

\subsection{Art der Arbeit}
%--------------------------
Um die Art derArbeit festzulegen gibt es folgende Optionsmöglichkeiten:\\

\begin{tabular}{lll}
\bf Option&\bf Art&\bf Titel der Arbeit \\
\hline\hline
  bachelor			& Bachelorthesis & Bachelorarbeit   \\
  master			& Masterthesis   & Masterthesis     \\
  diplom			& Diplomarbeit	 & Diplomarbeit 	\\
\end{tabular}\\[5mm]

Wird keine der oben aufgeführten Optionen angegeben wird Standardmäßig als Titel "`Wissenschaftliche Arbeit"' gewählt.


\subsection{Betreuer}
%--------------------
Für die/den jeweiligen BetreuerIn der \abk{WHZ} gibt es eine eigene Option. Die jeweilige Option ist dabei der kleingeschriebene Familienname
(\textit{golubski, beier, haeber, remke, lenk, krauss, schwarz}). 
Ist für die/den BetreuerIn der Arbeit innerhalb der  \abk{WHZ} keine Option vorhanden kann der Name über den Befehl \textit{$\setminus$betreuerProf\{name\}} nachträglich hinzugefügt werden. 
Für die/den BetreuerIn innerhalb des Unternehmens existiert der optionale Befehl \textit{$\setminus$betreuerUnternehmen\{name\}}.

\subsection{Druckfreundliche Version}
%------------------------------------
Für die Druckversion der Arbeit gibt es die Option \textit{printfriendly}. 
Diese entfernt die farbliche Hinterlegung der externen und internen Referenzen.

\subsection{Verzeichnisse}
%------------------------------------
Das Anzeigen der jeweiligen Verzeichnisse wird ebenfalls über Optionen der Dokumentenklasse gesteuert. 
Die Nachfolgende Auflistung zeigt die möglichen Optionen der anzeigbaren Verzeichnisse.\\

\begin{itemize}

  \item \textit{glossar}: Glossar und Abkürzungsverzeichnis
  \item \textit{abbverz}: Abbildungsverzeichnis
  \item \textit{tabverz}: Tabellenverzeichnis
  \item \textit{lstverz}: Listingverzeichnis (Codelistings)

\end{itemize}


		%,selbsterkl		% Selbständigkeitserklärung anzeigen
        %,kurzfassung	% Kurzfassung


\subsubsection{Variable}

Die Dokumentenklasse ist für verschiedene Arten von Arbeiten vorgesehen, die sich nur im Aufbau 
der Titelseiten unterscheiden. 
Abhängig vom gewählten Dokumententyp sind unterschiedliche Elemente für die Titelseiten erforderlich (siehe Tabelle \ref{tab:TitelElemente}).
Folgende Basisangaben sind für \textbf{alle} Arten von Arbeiten
erforderlich:
%
\begin{itemize}
\item[] %
\verb!\title{!\texttt{\em Titel der Arbeit}\verb!}! \newline%
\verb!\author{!\texttt{\em Autor}\verb!}! \newline%
\verb!\studiengang{!\texttt{\em Studiengang}\verb!}! \newline%
\verb!\studienort{!\texttt{\em Studienort}\verb!}! \newline%
\verb!\abgabemonat{!\texttt{\em Monat der Abgabe}\verb!}! \newline%
\verb!\abgabejahr{!\texttt{\em Jahr der Abgabe}\verb!}!
\end{itemize}
%



