\chapter{Einleitung}
\label{chap:einleitung}


\section{Ziel der Arbeit}
\label{sec:ziel}

%\section{Zielsetzung}
Mit der Bachelorarbeit wird das Studium abgeschlossen. Der Student weist darin
nach, dass er das während seines Studiums erworbene Wissen in einem größeren
Informatik-Projekt anwenden kann. Den wesentlichen Teil der Bachelorarbeit bildet
die schriftliche Dokumentation der in diesem Projekt angewendeten Methoden und
der erzielten Ergebnisse.
Die Arbeit soll einen folgerichtigen und in sich abgeschlossenen Aufbau, eine straffe
Gliederung, wissenschaftliche Exaktheit, kurze, sachliche und stilistisch einwand-
freie Ausdrucksweise und gedrängte Darstellung des Stoffes unter Hervorhebung des
Wesentlichen aufweisen. Alle Ausführungen sind in unpersönlicher Form zu fassen
(Ich-Form vermeiden).
Der Autor sollte sich an den \acs{DIN}-Normen für wissenschaftlich-technische Veröffentlichungen 
\cite{DIN05} orientieren. Für weiterführende Hinweise zur Anfertigung wissen-
schaftlicher Arbeiten siehe \cite{Rec06,Bri07,DLLS05}.

\section{Gliederung der Arbeit}
\label{sec:gliederung}

Gliederung der Arbeit
Bachelorarbeiten sollten nicht mehr als 40 Seiten umfassen. Es sind zwei gebundene
Exemplare, jeweils mit einer eingebundenen elektronischen Version der Arbeit auf
\acs{CD} oder \acs{DVD}, an der Westsächsischen Hochschule Zwickau im Dekanat der Fakultät
Physikalische Technik / Informatik abzugeben. Exemplare für den Betrieb bzw. das
Unternehmen sind gesondert zu vereinbaren.
Auf dem Rücken des Einbandes ist eine haltbare Rückenbeschriftung anzubringen,
die Kurzthema, Autor und Jahr der Einreichung enthält. 
Die Gliederung der Arbeit stellt die Inhaltsübrsicht des bearbeiteten Themas dar
und gibt Hinweise auf die vom Verfasser gesetzten Schwerpunkte.
Die Tiefe der Gliederung und Länge der einzelnen Abschnitte sollte nicht zu stark
variieren. Die Bestandteile der Arbeit sind in folgender Reihenfolge anzuordnen:\\

\begin{enumerate}
	\item Titelblatt (entsprechend Anlage),
    \item Autorenreferat als Kurzreferat (maximal 20 Zeilen),
    \item Angaben zur betreuenden Einrichtung, evtl. Danksagung
    \item Inhaltsverzeichnis: Abschnitte und zugehörige Seitenzahlen,
    \item Abkürzungsverzeichnis:
    \item Erklärung aller in der Arbeit verwendeten Abkürzungen in alphabetischer Reihenfolge,
    \item evtl. Abbildungs- Tabellen- und Anlagenverzeichnisse,
    \item Einleitung: Aufgabenstellung, Arbeitsziel, Einordnung in wissenschaftlichen oder praktischen  Kontext, Angaben zur Vorgehensweise und zum Aufbau der Arbeit, insbesondere des Hauptteiles,
    \item Hauptteil (in der Regel mehrere Kapitel, siehe Abschnitt 3): enthält theoretische und praktische Grundlagen, Dokumentation der verwendeten Methoden, Ergebnisse, Interpretation der Resultate,
    \item Zusammenfassung und Ausblick: wichtigste Ergebnisse der Arbeit, offene Fragen,
    \item Quellenverzeichnis (\nref{sec:quellenverz}),
    \item evtl. Anlagen (z.B. Inhalt der beigefügten \acs{CD} oder \acs{DVD}).
\end{enumerate}



\section{Hauptteil}
\label{sec:hauptteil}

Der Text ist in 12pt-Schrift auf Format A4, einseitig bedruckt, Randabstand links
35 mm und rechts 15 mm, auszuführen. In der Regel ist hierfür ein geeignetes Text-
satzsystem (mit Unterstützung zu Rechtschreibung, Silbentrennung, Formatierung,
Nummerierung, Indizierung) zu verwenden.
Formeln sind mit fortlaufenden arabischen Zahlen in runden Klammern zu num-
merieren, abschließende Klammer etwa an rechter Fluchtlinie. Klammern, Wurzeln 
sind in der erforderlichen Größe aufzuführen, Indizes eindeutig unterscheidbar.
Bei Hinweisen auf vorhergehende Textstellen und Gleichungen in der Arbeit sind die
Seitenzahl bzw. die Gleichungsnummer und die Seitenzahl anzugeben.
Abbildungen und Tabellen sind jeweils getrennt fortlaufend mit arabischen Zahlen zu
nummerieren und mit Unterschriften zu versehen. Bei Übernahme von Abbildungen
und Tabellen aus Literaturstellen oder sonstigen zitierbaren Quellen ist die Quelle
nach der Unterschrift zu vermerken. Befinden sich Abbildungen oder Tabellen nicht
auf der Seite, auf der im Text auf sie Bezug genommen wird, ist neben der Nummer
die entsprechende Seite im Text zu nennen. Die Größe der Abbildungen sollte stets
so gewählt werden, dass einerseits die inhaltliche Aussage gut erkennbar ist und
andererseits der gedrängten Darstellung entsprochen wird.
Anlagen größer A4 sind als Kopien gefaltet in die Arbeiten einzuordnen.
Ubernahmen von Text (auch sinngem.), Formeln, Software, Abbildungen, Tabellen
usw. aus fremden Quellen sind zur Gewährleistung des Urheberrechts mit einer
in eckigen Klammern gesetzten Abkürzung von Autor und Erscheinungsjahr der
Quelle zu kennzeichnen. Zu dieser Abkürzung ist im Quellenverzeichnis die Quelle
anzugeben. Wörtliche Wiedergaben sind in Anführungszeichen zu setzen.


\section{Quellenverzeichnis}
\label{sec:quellenverz}

Im Quellenverzeichnis muss jedes benutzte Dokument aufgeführt sein. Jeder Leser, 
insbesondere jeder Gutachter, muss alle Quellen jederzeit nachprüfen können.
Aus diesem Grund sind dynamische“ Websites nicht als Quellen geeignet. Die
Zulässigkeit der Angabe von Wikipedia-Seiten als Quelle ist mit dem Betreuer abzu-
stimmen. Es wird empfohlen, die dort angegebenen Quellen direkt zu konsultieren.
Auch beim Zitieren von \acs{URL}s müssen alle ublichen bibliografischen Angaben vorhanden 
sein: Name (bei Dokumentationen oder Spezifikationen eventuell auch eine
Versionsnummer), Autor, Datum und Ort (der \acs{URL} liefert nur den Ort). Mehr zum
Thema Zitieren von Quellen aus dem Internet ist in \cite{RS02} und \cite{Tap96} zu finden.


\section{Thesen}
\label{sec:thesen}

Die Thesen sollen in kurzer Form (pro These 1 bis 2 Sätze) die wichtigsten eigenen
Beiträge zur Lösung der Aufgabenstellung der Arbeit sowie die sich daraus ergeben-
den Aussagen enthalten. Die Thesen sind einmal in jede Bachelorarbeit einzubinden,
zusätzlich sind sie 10fach mit den beiden Exemplaren abzugeben.
