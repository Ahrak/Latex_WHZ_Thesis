\chapter[Hinweise f�r \emph{Word}-Benutzer]{Hinweise f�r \emph{Word}-Benutzer%
\protect\footnote{Dieses Kapitel ist ein Relikt aus fr�heren Versionen, die umfangreiche Hinweise f�r die Erstellung von Diplomarbeiten mit \emph{Word} enthielten.}%
}
\label{chap:Word}

Wie bereits in der Einleitung erw�hnt, ist \emph{Word} f�r das
Schreiben von umfangreicheren Werken wie B�cher und
Diplomschriften nur bedingt geeignet. \emph{Word} besitzt zwar
einen immens gro�en Funktionsumfang,  manche einfach anmutende
Aufgaben erfordern aber bisweilen sehr umst�ndliche Ma�nahmen oder
sind schlichtweg unm�glich. Eine unangenehme Eigenschaft ist
weiters, dass Word-Dokumente gelegentlich in fehlerhafte Zust�nde
geraten k�nnen, die man nur mehr durch R�ckkehr zu einer vorher
gesicherten Version (sofern vorhanden) reparieren kann. 
Tats�chlich scheinen sich bei den neueren
(XML-basierten) Office-Versionen einige der bisherigen
Schwierigkeiten noch verst�rkt zu haben, \zB\ die noch
umst�ndlichere (und empfindliche) Verwaltung von Formatvorlagen.

Das soll nicht hei�en, dass mit entsprechender Disziplin und Detailkenntnis nicht auch mit \emph{Word} gro�e Dokumente sauber und erfolgreich hergestellt werden k�nnen, wie auch die Produkte mancher Sachbuchverlage zeigen.
Ich pers�nlich w�rde aber dazu nicht ermutigen und habe daher die in diesem Kapitel fr�her zusammengefassten Hinweise f�r den Umgang mit \emph{Word} entfernt.


Falls man \emph{Word} ohnehin
nur oberfl�chlich beherrscht, sollte man daher �berlegen, ob man es
nicht gleich mit \latex versuchen m�chte. 
Bei durchaus vergleichbarem Lernaufwand wird sich wahrscheinlich
das Ausma� an Frustration -- mit Sicherheit aber das Ergebnis -- 
deutlich unterscheiden.
Falls man von Word 
auf \latex umzusteigen m�chte und zu diesem Zeitpunkt bereits
umfangreiches Material in \emph{Word} vorhanden ist, sollte man sich das
Programm \texttt{rtf2latex}%
\footnote{\zB \url{www.tex.ac.uk/tex-archive/support/rtf2latex2e/}}
ansehen, das \emph{Rich Text Format} (RTF) in \latex-Dateien �bersetzt.


Als professionelle WYSIWYG-Alternative bietet sich \zB\ \emph{Indesign} von \emph{Adobe} an, das f�r den Schriftsatz angeblich �hnliche Algorithmen wie \latex\ verwendet. Bez�glich mathematischer Elemente, gleitender Platzierung von Abbildungen und Tabellen, sowie der Verwaltung von Literaturangaben kommt Indesign allerdings an \latex\ derzeit nicht heran.
